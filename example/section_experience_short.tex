% Awesome Source CV LaTeX Template
%
% This template has been downloaded from:
% https://github.com/darwiin/awesome-neue-latex-cv
%
% Author:
% Christophe Roger
%
% Template license:
% CC BY-SA 4.0 (https://creativecommons.org/licenses/by-sa/4.0/)

%Section: Work Experience at the top
\sectionTitle{Professional Experience}{\faSuitcase}
%\renewcommand{\labelitemi}{$\bullet$}
\begin{experiences}
  \experience
    {Present}   {Research Assistant Professor}{University of Washington}{Seattle, WA}
    {November 2021} {
                      \begin{itemize}
                        \item Research in the division of healthcare simulation sciences, department of surgery
                        \item Investigated reinforcement learning AI agent actions on a simulated sepsis patient
                        \item Quantified curvature to define injury pathology in hemorrhage
                        \item Identified Naloxone requirements for Fentanyl overdose on a virtual patient population 
                        \item Created a database for experimental tissue mechanical properties data connected to a website using the Django framework
                        \item Collaborated with clinicians the department of emergency medicine and anesthesiology to investigate trauma physiology
                        \item Developed the build system for the MoHSES healthcare simulation framework using CMake and Buildbot.
                        \item Led four student teams in the department of biomedical engineering
                        \item Created all course content for the computational physiology course in the masters program on healthcare simulation research
                        \item Wrote grant proposals totaling ~ 300K dollars over two years, supporting private industry and clinicians across the school of medicine
                        \item Presented research at SIAM dynamical systems, SIAM life sciences, and American College of Surgeons Simulation Summing 
                        \item Organized and led two symposiums on mathematical models of immunology and AI in medicine  
                      \end{itemize}
                    }
                    {}
  \emptySeparator
  \experience
    {October 2021} {Biomedical Modeling Group Leader | Senior Engineer}{Applied Research Associates, Inc.}{Raleigh, NC}
    {February 2016}{
                      \begin{itemize}
                        \item Secured and managed \$6 million in research and development funds through grants from the Defense Health Agency and Army Research Lab
                        \item Principal Investigator working with a multidisciplinary team across 4 different multi-million dollar projects    
                        \item Promoted three times over the course of five and a half years of service                       
                        \item In charge of agile development processes, product roadmap, delivery scheduling, risk management and direct communication with government customers
                        \item Principal investigator of BioGears, BurnCare training application, and the traumatic brain injury angiogensis projects                
                        \item Organized teaming across three research hospitals and multiple small businesses
                        \item Communicated research progress through multiple conferences and peer reviewed publications, including the BioGears 2020 conference
                        \item Oversaw implementation and research of all models associated with BioGears releases 6.3-8.0                                                   
                      \end{itemize}
                    }
                    {}
  \emptySeparator
  \experience
    {August 2015}     {Visiting Assistant Professor}{Duke University}{Durham, NC}
    {August 2014}{
                      \begin{itemize}
                        \item Analyzed how pressure changes induced by heart failure affect the hemodynamic and re-absorption function of the kidney
                        \item Taught two semesters of introduction to partial and ordinary differential equations, developed all course materials                
                        \item Coordinated validation and experimental data with University of Ontario research hospital clinicians                   
                        \item Developed a fluid-structure interaction model studying forces on red blood cells 
                      \end{itemize}
                    }
                    {}

\end{experiences}
